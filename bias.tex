\documentclass[11pt, oneside]{article}
%\usepackage{ijcai18}
%\usepackage{apacite}
\usepackage{geometry}
\geometry{letterpaper}
\usepackage{graphicx}
\usepackage{color}
\usepackage{xcolor}
\usepackage{amssymb}
\usepackage{hyperref}
\title{NIPS 2018 Competition proposal: Bias in Face-based Attributes Analysis
 as a function of skin type}
\author{Esube Bekele\thanks{The Leader organizer should be the first author of
		 the proposal.} \and Joy Buolamwini \and Timnit Gebru \and Wallace
	 Lawson
    %\and Ian Goodfellow
    %\and Samy Bengio
    \\
{\tt esube.bekele.ctr@nrl.navy.mil}\\
}
\date{\today}

\makeatletter
    \let\@internalcite\cite
    \def\cite{\def\citeauthoryear##1##2{##1, ##2}\@internalcite}
    \def\shortcite{\def\citeauthoryear##1{##2}\@internalcite}
    \def\@biblabel#1{\def\citeauthoryear##1##2{##1, ##2}[#1]\hfill}
\makeatother

\begin{document}
\maketitle

\subsection{Overview of the competition}
% Summarize the background, available data, methods, available baseline,
% potential impact.
% Background and impact review

Despite recent success in face recognition and facial attributes analysis,
research suggests the existence of a wide range of unintentional biases towards
a specific racial or gender group \cite{phillips2011other} \cite{klare2012face}.
Facial analysis is increasingly becoming pervasive in our daily lives for
various applications such as face-based verification, identification, and
criminal suspect profiling.  These biases have significant ramifications as
these automated facial recognition and facial attributes analysis systems are
widely adopted by law enforcement and other critical areas such as healthcare.
Most common facial attributes in automated facial analysis include gender,
race, and age \cite{fu2014learning} \cite{ng2015review}
\cite{han2015demographic}. Face-based Gender and facial skin tone
classification are covered in this competition.

% Availbale data review
A recent study on effect of demographics on facial gender classification
showed this bias on three commercially available gender classification systems
\cite{buolamwini2018gender}. As part of eliciting these biases, a new dataset
was collected that is balanced by gender and skin tone type called the Pilot
Parliaments Benchmark (PPB). Although there are many publicly available face
datasets that have race, gender, age, and other face-based attributes, we will
primarily use this benchmark dataset for evaluation (as the test set) in this
competition as attempts were made to balance the dataset based on demographic
groups. This pilot analysis showed that three commercial face-based gender
recognition systems by Microsoft, Face++, and IBM resulted in error rates of
23.8\%, 36.0\%, and 31.1\%, respectively, in dark skin type female samples from
the PPB dataset. In contrast, the three systems had an average error rate of
0.53\% for light skinned color males despite the fact that only 30.3\% of the
benchmark were light skinned males\cite{buolamwini2018gender}.

% Methods and available baseline for each track
The competition will have two separate tracks. In the first track, researchers
will be generating samples from existing datasets to illustrate such racial and
gender biases in existing commercial systems and
methods for analyzing facial attributes. In the second track, we will provide a dataset that is not balanced by race and
gender, and researchers will submit solutions that learn facial attributes
while minimizing disparities by race and gender in the classification performance.
We chose gender classification for two reasons. First, the original PPB dataset
and pilot audit of the commercial systems is done on gender recognition and it
will be easier for comparisons of the results coming out of this competition
with the pilot study. Secondly, gender recognition involves recognition of
several facial attributes such as facial shape, hair style and certain facial
features and hence gender recognition implicitly is comprehensive.

{\bf Contributions}: The two tracks of this competition have the potential to 
spur new research lines or new application of existing methodology in 
alleviating unintended algorithmic bias due to imbalanced datasets specifically 
for face-based gender and other attributes recognition. 
\begin{itemize}
	\item The first track is aimed at auditing algorithmic bias in commercial 
	facial analysis systems and pinpointing the source of the bias 
	(specifically if the bias is solely due to skin tone). Skin tone 
	manipulation is a specific instance of image morphing in the color space. 
	This could potenially attract solutions involving both traditional skin 
	segmentation and manipulation \cite{martinson2013identifying} and recent 
	generative (and/or adversarial) 
	methods for skin tone manipulation.
	\item The second track is aimed at developing solutions that would learn 
	facial attributes from natural, imbalanced and unconstrained datasets. As 
	facial and full-body attributes recognition \cite{bekele2017multi} becomes 
	increasingly important
	for soft-biometrics, watch-list surveillance  \cite{kamgar2011toward}, and
	person of interest identification and re-identification 
	\cite{best2014unconstrained}, predictions that are equitable towards all 
	demographic groups are certainly crucial. We hope to see several 
	algorithmic solutions including, but not limited to, sample loss weighting, 
	architectural innovations for skewed datasets, probability calibrations, 
	few-shot or single-shot recognition, and learning the tail distribution.
	\item Most importantly, the competition has the potential to gather 
	practitioners from several cross-disciplines such as machine learning, 
	ethics and accountability in AI, biometrics, sociology, and neuroscience.
\end{itemize}


\subsection{Keywords}
% Competition keywords, up to five, from generic to specific.
Other-race Effect, Automated Face Analysis, Gender Recognition, Bias in Facial
attribute Analysis

\subsection{Novelty}
% (Have you heard about similar competitions in the past? If yes, describe the 
%key differences. O/w disregard)
%Indicate whether this is a completely new competition, a competition part of a 
%series, eventually re-using old data.
To the best of our knowledge this will be the first time such competition that 
is targeted at eliciting racial and gender biases in face recognition and 
analysis systems and attempting to solve these biases using techniques other 
than simply balancing the training dataset. Balancing datasets for every 
demographic group and gender is usually impractical for commercial applications 
with consequential impact such as watch-list surveillance 
\cite{kamgar2011toward} and person of interest identification 
\cite{best2014unconstrained}.

\section{Competition description}

\subsection{Background and impact}

% Provide some background on the problem approached by the competition and 
%fields of research involved. Describe the scope and indicate the anticipated 
%impact of the competition prepared (economical, humanitarian, societal, etc.). 
%Justify the relevance of the problem to the targeted NIPS community and 
%indicate whether it is of interest to a large audience or limited to a small 
%number of domain experts (estimate the number of participants). A good 
%consequence for a competition is to learn something new by answering a 
%scientific question or make a significant technical advance.

% Extended Background
In psychology, there is a well documented phenomenon called "own-race" or 
"other-race" bias that people recognize faces and face related attributes of 
their own race better than that of other races \cite{furl2002face}. This 
inherent bias manifests itself in automated face recognition and analysis 
systems in part due to the unintentional imbalance in the datasets used to 
train such systems \cite{phillips2011other}. As shown in the Face Recognition 
Vendor Test (FRVT) by the National Institute of Standards and Technologies 
(NIST) evaluated every four years, the performance of automated facial analysis 
has been steadily improving \cite{grother2010report}. The first breakdown of 
facial analysis systems performance by demographic groups 
\cite{phillips2011other} showed, however, that this performance improvement is 
not uniform across demographic groups of race and gender.

% Extended impact
With the wide spread adoption of automated facial analysis systems by law 
enforcement such as state and local police and federal immigration enforcement, 
proper regulation of such systems is of paramount importance. A recent report 
by the Georgetown law school center on privacy and technology highlights the 
presence and consequences of racial bias embedded in such systems 
\cite{garvie2016perpetual}. The report indicated that 50\% (and expanding fast) 
of American adults are in a law enforcement face recognition network. It also 
highlights the dire consequences of such unregulated law enforcement face 
recognition showing such systems disproportionately affect African Americans as 
these systems are less accurate in identifying African Americans. Recent 
studies show the effect demographic groups on face analysis performance 
\cite{han2015demographic} \cite{farinella2012demographic} \cite{klare2012face}. 
Although these preliminary findings seem to suggest the existence of such bias 
and its potential consequences, there is dearth of research in this area to 
both find the source of such bias in facial analysis systems and solutions.

% About the competition and relevant experience by authors
It is important to pinpoint whether such bias is primarily due to skin tone 
(race or ethnicity) and gender or these systems are exploiting other background 
information such as clothing and other visible attributes that correlate with 
specific demographic groups. It is also important to solve this bias in facial 
analysis systems in the presence of an imbalanced training dataset. Therefore, 
this competition has two tracks correspondingly.

The first track requires participants to use the PPB dataset to generate 
samples that elucidate the source of such bias in commercial face recognition 
software.  To generate these samples, competitors will synthetically manipulate 
the demographic dimensions of gender and facial skin tone of faces in the PBP 
dataset. The original PPB dataset is balanced with respect to these 
demographics groups to avoid additional bias in the testing set 
\cite{buolamwini2018gender}.

In the second track, we solicit submissions that focus on a combination of 
novel and existing algorithms and architecture that result in relatively 
uniform performance across demographic groups. For this track, we will add 
additional samples as an extended test set to be used in combination with the 
PPB dataset. Esube has experience on data science competitions (with a master 
rank on kaggle.com) and we intend to use kaggle.com for hosting the 
competition. Joy and Timnit has collected and annotated the original PPB 
dataset.

\subsection{Data}
%If the competition uses an evaluation based on the analysis of data,
%please provide detailed information of the available data and
%their annotations, and, in case, what the data generation
%procedure will be (in this case, it must be clear in the document
%that the data will be ready prior to the official launch of the
%competition). Please justify that: (1) you have access to large
%enough datasets to make the competition interesting and draw
%conclusive
% results; (2) the data can be made freely available;(3) the ground truth has 
%been kept confidential.

Demographic imbalances in existing face-based gender recognition datasets led 
to creating a new dataset for gender classification as evaluation benchmark 
called the Pilot Parliaments Benchmark (PPB) \cite{buolamwini2018gender}. This 
dataset will be primarily used as an evaluation set for the preliminary rounds 
of the competition. The dataset is composed of 1270 identities (1 
image/subject) from 3 African (Rwanda, Senegal, and South Africa) and 3 
European countries'(Iceland, Finland, Sweden) national parliaments These 
countries were chosen by their ranking on percentage of women representation in 
their national parliaments. Fig. \ref{fig_ppb} shows representative samples and 
average images from each country and gender. Table \ref{ppb} summarizes the PPB 
dataset statistics compared to recent face recognition benchmarks.

\begin{figure}[t]
    \label{fig_ppb}
    \centering
    \includegraphics[width=140mm]{fig/ppb}
    \caption{Sample images and average faces from each country and gender in 
    	the PPB dataset.}
\end{figure}

\begin{table}[b]
    \caption{Dataset statistics of PPB compared to IJB-A and Adience face recognition benchmarks}
    \label{ppb}
    \centering
    \begin{tabular}{cccc}
        
        \hline
        feature & PPB & IJB-A & Adience \\
        \hline
        Release Year & 2017 & 2015 & 2014 \\
        \# of Subjects & 1270 & 500 & 2284 \\
        Avg. IPD (in pixels) & 63 & - & - \\
        BBox Size & 141 (avg) & $\geq$ 36 & - \\
        Image Width & 160-590 & - & 816 \\
        Image Height & 213-886 & - & 816 \\      
        \hline
    \end{tabular}
\end{table}


The major defining characteristics of facial analysis and recognition benchmark 
datasets is their imbalance with respect to racial/ethnic and gender 
demographic groups \cite{phillips2011other, han2015demographic}. In the PPB 
dataset, intentional effort was made to balance among these demographic groups 
to allow a balanced evaluation of facial analysis algorithms (see Table 
\ref{skin_ton}). The countries included in the dataset were selected because 
they were among the top 10 countries based on their percentage of women 
representation in their national parliaments. This balance in the test dataset 
(PPB) sets equivalence across the demographic groups in all the metrics 
evaluated.

\begin{table}[b]
    \caption{Facial skin ton and gender breakdown of the dataset PPB compared to IJB-A and Adience}
    \label{skin_ton}
    \centering
    \begin{tabular}{cccc}
        
        \hline
        Demographics Group & PPB & IJB-A & Adience \\
        \hline
        Darker Female & 21.3\% & 4.4\% & 7.4\% \\ 
        Darker Male & 25.0\% & 16.0\% & 6.4\% \\
        Lighter Female & 23.3\% & 20.2\% & 44.6\% \\
        Lighter Male & 30.3\% & 59.4\% & 41.6\% \\       
        \hline
    \end{tabular}
\end{table}

For the second track of the competition, we will use the CelebA dataset 
\cite{liu2015deep} as a training set and the PPB dataset for preliminary 
evaluation on gender classification. While CelebA is a large unconstrained 
images of faces in the wild, PPB is a more balanced and constrained, in terms 
of pose and illumination, evaluation benchmark. CelebA contains 10,000 
identities with 20 images each on average for a total of 202,599 images labeled 
with 40 attributes and 5 key points. The attributes include gender and skin 
color (only for pale skin) and hair color. Hence, it represents a typical 
imbalanced and unconstrained training set for facial analysis.

For the first track, we will release half of the African and half of the 
European set of images for a total of 634 images as part of the development 
package to generate morphed images by manipulating the skin tone and gender 
dimensions. We reserve the remaining 636 images for the second track again 
released as part of the development package. We are extending this dataset by 
collecting and annotating 2500 additional parliamentarians' images from 
countries that are not already included in the original PPB and we will 
sequester this set as the final round test set for both tracks of the 
competition.

\subsection{Tasks and application scenarios}

%Describe the tasks of the competition and explain to which specific real-world 
%scenario(s) they correspond to. If the competition does not lend itself
%to real-world scenarios, provide a justification. Justify that the problem 
%posed are scientifically or technically challenging but not impossible to
%solve. If data are used, think of illustrating the same scientific problem 
%using several datasets from various application domains.

This competition will have two separate tasks that are divided into two 
separate tracks. Competition participants are free to submit to both tracks.

\subsubsection{Commercial Face Analysis Systems Challenge}
In this challenge, participants are required to evaluate the performance of 
three commercial face-based gender classification systems (Microsoft, IBM, and 
Face++) with respect to the skin tone continuum. The main purpose of this 
challenge is to show the effect of skin type on gender classification. This is 
analogous to generating adversarial examples but constrained only by changing 
only the skin type. This challenge helps to pinpoint whether demographics 
(specifically skin type in this case) alone influence performance of gender 
classifiers and to rule out other confounding factors such as visible portions 
of clothing that would correlate with these specific demographic groups.

In this track, each participant will be given 634 images from the PPB dataset 
as part of a development kit that also contains tools for cross validation, 
pre-trained gender classifiers, links to the commercial face analysis 
cloud-based tools and evaluation performance metrics. For each image in the 
development package, each participant is expected to generate new image that is 
similar to the original image in identity. The only modification participants 
are allowed to make to the input images is to manipulate the skin tone of the 
subject in the image using any face morphing or feature-level facial attribute 
manipulation models.

The final performance in this challenge will be evaluated with a separate test 
set of images from parliamentarians from other countries to minimize 
over-fitting. In this task participants are prohibited to perform any 
adversarial manipulation on the image to get lower score other than the skin 
tone manipulation. Submission that results in the lowest performance in metrics 
as described in the metrics section will be winners.

\subsubsection{Face-based Gender Recognition Challenge}
This second task is aimed at training gender recognition models with an 
unconstrained non-balanced realistic dataset producing a balanced performance 
in each of the four course demographic categories. For this purpose, each 
participant will be required to train a model (or ensemble of models) on CelebA 
dataset. Participants are allowed to use all the 39 attribute labels as long as 
it helps improve the gender classification. For instance, they could use the 
"Pale\_skin" attribute to categorize the images based on skin lightness and use 
sample weights during training. Participants are allowed to employ any 
architecture, training technique, data augmentation (no external dataset is 
allowed), output probability calibration and other methods for training on 
imbalanced dataset.

We will release 636 images of the PPB dataset with the development kit for 
local cross validation and testing. We will collect and annotate new set of 
final test set images of parliamentarians from other countries than the 
original six countries in the original PPB dataset to avoid over-fitting.

In both tracks of the competition, participants are required to submit their 
code as part of their submission by the deadline date to be considered for the 
first three top position.


\subsection{Metrics}
\label{sec:metrics}
%For quantitative evaluations, select a scoring metric and justify
%that it effectively assesses the efficacy of solving the problem
%at hand. It should be possible to evaluate the results
%objectively. If no metrics are used, explain how the evaluation
%will be carried out.
We will use the metrics mean accuracy (see Eqn. \ref{eqn:ma}) and F1 score (see 
Eqn. \ref{eqn:f1}) as evaluation metrics for this competition mainly. Mean 
accuracy is more informative than regular accuracy (or error) metric on the 
balance of the accuracy in both values of the gender class (i.e. male and 
female). F1 score balances the precision (see Eqn. \ref{eqn:prec}) and recall 
(see Eqn. \ref{eqn:rec}) and hence could serve as a more balanced performance 
metric than just either precision or recall.

\begin{equation}
\label{eqn:ma}
mA =   \frac{|TP|}{|P|} + \frac{|TN|}{|N|}
\end{equation}

where $TP$ and $TN$ are the true positive and true negatives over total 
positive samples and total negative samples of the gender attribute.

\begin{equation}
\label{eqn:prec}
Prec =  \frac{1}{N}  \sum_{i=1}^{N}   \frac{|Y_{i} \cap f(x_{i}) |}{|f(x_{i}|}
\end{equation}

\begin{equation}
\label{eqn:rec}
Recall =  \frac{1}{N}  \sum_{i=1}^{N}   \frac{|Y_{i} \cap f(x_{i}) |}{|Y_{i}|}
\end{equation}

\begin{equation}
\label{eqn:f1}
F1\ score =  \frac{2.Prec.Recall}{Prec+Recall}
\end{equation}

where $N$ is the total number of samples considered at one time and $|.|$ is 
the set cardinality.

For the first vendor evaluation track of the competition, we will re-score the 
updated vendors' systems in terms of F1 score (see Eqn. \ref{eqn:f1}) using the 
original PPB dataset for the four demographic groups (dark skin female, dark 
skin male, light skin female, and light skin male) and will average the F1 
scores as baseline. Images will be generated by running the source code of each 
submission on the evaluation set.  The evaluation set will be used as testing 
benchmark to score the three vendor systems (Microsoft, IBM and Face++).  This 
will be evaluated on each of the four demographic groups using both mean 
accuracy and F1 score, the lowest average F1 score across the three vendors and 
the four demographic groups will determine the winners. If there is close to 
tie on the average F1 score, mean accuracy will be used to separate the teams.

For the second balanced gender recognition challenge, we will generate 4 F1 
scores for the four demographic groups for each submission on the sequestered 
extended PPB benchmark. We will then sort submissions based on average F1 
scores across the four groups. We will cut off the top 10 submissions (models) 
that have the higher average F1 scores. Although F1 score is a more balanced 
performance metric, averaging across the four demographic groups could dilute 
the 'other-race' effect. Hence we will perform Freidman's ANOVA test on the 
difference between the probability submitted for each sample in each group and 
the ground truth, i.e. $|y_{i} - f(x_{i})|$ for the $ith$ sample. The test will 
produce the probability that there is a difference across the four demographic 
groups, $p-value$ and $\chi^{2}$ statistics that shows the consistency of 
prediction errors across the four groups. We will use this $\chi^{2}$ 
statistics to re-sort the top 10 teams for consistency and this will determine 
the winners for the second part of the challenge.

\subsection{Baselines and code available}

%Specify what are (will be) the baselines for the competition, and
%whether there is available code for participants (e.g., a starting
%kit) and evaluation. Provide preliminary results, if available.
We will release a self-contained development kit that contains, the necessary 
datasets needed for each track (see Section \ref{sec:protocol}), the 
implementation of the evaluation metrics described in the Metrics section (see 
Section \ref{sec:metrics}), and the baseline results.

For the commercial vendors audit section, the baseline will be the preliminary 
results presented in \cite{buolamwini2018gender}. After this preliminary study 
two of the evaluated vendors (Microsoft and IBM) updated their gender 
recognition systems for better performance in the dark skin female category. We 
will evaluate the new systems and update the baselines when we release the 
development kit.

For the balanced gender classification task, our baseline model is ResNet50 
pre-trained on ImageNet and fine-tuned on CelebA for gender recognition as part 
of the development kit. We will train the model, evaluate it on PPB based on 
the split explained in the Protocol section (see Section \ref{sec:protocol}) 
and release the results as the baseline benchmark within the development kit. 
We will also release a pytorch-based code for training such a model on the 
CelebA dataset for gender recognition.

\subsection{Tutorial and documentation}

%Provide a reference to a white paper you wrote describing the
%problem and/or explain what tutorial material you will provide.
Participants will be encouraged to read \cite{buolamwini2018gender} and the 
paper will be included in the development package as a starting point to 
understand the problem. Then, we will release extensive documentation and 
tutorial examples for how to evaluate the vendor benchmarks once participants 
have generated the skin tone modified images using the method of their choice. 
we will also thoroughly document the gender recognition pytorch code using 
CelebA. We will add a tutorial example showing how to evaluate the trained 
model on the local preliminary round part of the PPB evaluation benchmark. The 
final test set of the extended PPB dataset that is currently being collected 
and annotated will NOT be released until the competition is over.

To summarize, the development package will contain the following:

\begin{itemize}
	\item The pilot study paper wrote by two of the authors (Joy and Timnit) of 
	this competition proposal that explains the problem 
	\cite{buolamwini2018gender}.
	\item Two splits of the PPB dataset as evaluation benchmarks for the 
	preliminary rounds of the two tracks of the competition.
	\item Tutorial and documentation on how to perform the vendor evaluation 
	once participants have generated an altered image for the first track of 
	the competition.
	\item Pytorch code to fine-tune an ImageNet trained ResNet50 (participants 
	would change this to their architectures and their techniques to combat 
	data inter-class imbalance), the trained model weight, and through 
	documentation
	\item Example tutorial that shows how to evaluate their trained documents 
	on the split set of the PPB dataset for their local cross-validation.
\end{itemize}

\section{Organizational aspects}
\subsection{Protocol}
\label{sec:protocol}
%Explain the procedure of the competition: what the participants will have to 
%do, what will be submitted (results or code), and the evaluation procedure.
%Will there be several phases? Will you use a competition platform with on-line 
%submissions and a leader board? Indicate means of preventing cheating.
%Provide your plan to organize beta tests of your protocol and/or platform.

Each track of the competition will have preliminary rounds of up to 5 
submissions maximum and a final round submission. The preliminary rounds are 
for the purpose of giving participants feedback other than their local internal 
validation using the preliminary test sets released together with the 
development kit. The final round will be scored on the private test set and 
only this final round of submissions are considered for ranking teams.

We plan to apply for hosting research based competition with kaggle.com. It has 
self contained on-line submission and leader boards.

\subsubsection{Preliminary Rounds}

For the commercial face-based gender recognition systems evaluation track, we 
will release 634 random images from the PPB split by countries for internal 
cross validation and preliminary rounds submissions. The protocol for the 
preliminary rounds of this track are as follows.

\begin{enumerate}
    \item Participants are expected to generate a new sample for each of the 
    634 images in the development kit and submit the generated image. 
    Participants are allowed to use their own image manipulation models or 
    traditional pipeline. No commercial software is allowed.
    \item Organizers will score the newly generated images on the three 
    commercial face-based gender recognition systems and performance metrics 
    for each vendor and average performance metrics will be released for each 
    of the participants. At this stage, organizers will not check for cheating.
\end{enumerate}

For the face-based gender recognition challenge, participants will follow the 
following protocols.

\begin{enumerate}
    \item Organizers will release 636 random images from the PPB evaluation 
    benchmark as part of the development kit.
    \item Participants are expected to train their face-based gender 
    recognition models on CelebA dataset. The aligned and cropped versions of 
    the CelebA dataset together with 40 attributes (gender is one of these 
    attributes) will be made available both on the competition site and 
    kaggle.com. We encourage submissions that exploit attributes other than 
    gender to help in balanced gender recognition in a multi-label 
    classification. Only the gender prediction will be used to score 
    submissions.
    \item Organizers will score the submissions on the 636 preliminary round 
    test sets and release the gender recognition performance to participants 
    (in the form of public leader board).
\end{enumerate}

\subsubsection{Final Rounds}
For the final round we will set aside 2500 images and their skin type and 
gender labels for scoring the final submissions. To prevent cheating in this 
final round, the following protocols are followed:

\begin{itemize}
    \item The labels for the final round images are held confidentially and 
    will not be made available until the competition ends and winners are 
    announced. The images will be released
    \item All participants will have to submit their code or open source it on 
    github and submit a link to it to be considered for the top 3 positions. 
    Submissions without source code in the final round will not be scored.
    \item For the first track of this competition, organizers will make at most 
    care to run the code and generate new samples for the 2500 final test 
    images. Special care will be taken for the top 3 teams to make sure that 
    their code is not generating any adversarial samples other than skin tone 
    manipulation in an effort to exploit adversarial vulnerability of these 
    commercial gender recognition systems. After the skin tone manipulation, 
    the original image and the generated image will not be $L_{\infty}$ 
    consistent and hence it is difficult to check if there is added adversarial 
    attack on the generated images.
    \item For the gender recognition challenge track, organizers will run the 
    classification code submitted by participants on the 2500 images and score 
    them against their labels.
\end{itemize}


In this final round, the protocol for each track is as follows.

\begin{enumerate}
    \item For both tracks of the competition, participants are required to 
    submit their code by the competition deadline.
    \item Organizers will run the source code on the 2500 final round test 
    images and generate manipulated images and evaluate the three commercial 
    systems with them in the case of the first track. For the second track, 
    organizers will run the code and generate gender predictions and score them 
    against the ground truth gender labels.
    \item Organizers will determine the top 3 teams in each track.
    \item For the first track, the top 3 submissions are selected based on 
    performance metrics for each vendor and average lowest performance metric 
    for each of the two demographic groups (i. e., dark skin type male and dark 
    skin type female).
    \item The source code from the top three teams will be carefully examined 
    to make sure adversarial attacks are not injected together with the skin 
    tone manipulation.
    \item For the second track, the top 3 teams are selected based on the 
    highest balanced performance metrics for all the four demographic groups.
    \item Organizers will announce the results and release the final round 2500 
    test images and their labels. Organizers will combine the new 2500 test 
    images with the original PPB dataset so that further research and 
    commercial systems will use this as evaluation benchmark test sets for 
    gender and race balanced facial analysis.
\end{enumerate}


\subsection{Rules}

%Provide a list of special rules.
%For qualitative evaluations (e.g. demonstration competitions), select a 
%committee and prepare guidelines for the committee.
\begin{enumerate}
    \item Anyone can participate with the exception of the organizers and 
    anyone that has any conflict of interest with the organizers.
    \item To be eligible for scoring in the final rounds, participants are 
    required to submit their source code or open source it on github.
    \item Source code must not produce any errors or the submission will be 
    abandoned without scoring.
    \item Participants are required to submit clear documentation of their code 
    on how to run it on the final round test images. The documentation also 
    should clearly list set of dependencies for the code.
    \item Any leader board probing will not be allowed and will be grounds for 
    elimination from the competition.
    \item External datasets are prohibited.  Only the CelebA dataset and PPB 
    dataset released by the organizers are allowed for this competition.
    \item Participants are allowed to pre-training with ImageNet only.  No 
    other datasets may be used.
\end{enumerate}

\subsection{Schedule}

%Provide a time line for competition preparation and for running the 
%competition itself. Propose a reasonable schedule leaving enough time for the 
%organizers
%to prepare the event (a few months), enough time for the participants to 
%develop their methods (e.g. 90 days), enough time for the organizers to review 
%the entries, analyze and publish the results.
The proposed competition schedule is as follows.

{\it April 1st, 2018} {\bf Start of competition promotion}: Start of 
Competition Promotion. We will launch the competition website with the call for 
submissions. Start heavy promotion of the competition via social media, at 
upcoming conferences and mailing lists.

{\it June 30th, 2018} {\bf Release of development kit and official start of the 
	competition}: The competition starts with the release of the development 
kit.

{\it June 30th, 2018 - October 1st, 2018} {\bf Duration of the competition}: 
Submissions are allowed starting this date until the competition closes for a 
maximum of 5 preliminary round of submission by each team for each track.

{\it October 1st, 2018} {\bf Deadline for the final round}: Each team will have 
one more final round submission which is due on October 1st, 2018.

{\it October 1st 2018 - October 30th, 2018} {\bf Final round evaluation}: We 
will evaluate final round submissions on the separate test images that are not 
publicly released.

{\it November 1st 2018} {\bf Announcement of winners}: Winners (top 3 teams) in 
each track will be announced and the final round test images will be made 
public.

\subsection{Competition promotion}

%The plan that organizers have to promote participation in the competition 
%(e.g., mailing lists in which the call will be distributed, invited talks, 
%etc.).
This competition will be promoted using the organizers' Facebook, Twitter, 
Reddit and other social media accounts and pages, and mailing lists. Moreover, 
we encourage black professionals in AI to take part in this competition. To 
this effect, we will heavily promote it in the Black in AI Facebook group and 
discussion forums and we wish winners of both tracks to present in the next 
Black in AI workshop which will be co-located with NIPS 2018.

\subsection{Organizing team}

%Provide a short biography of all team members, stressing their competence for
%their assignments in the competition organization. Make sure to include: 
%coordinators, data providers, platform administrators, baseline method 
%providers, beta testers, and evaluators.
{\bf Esube Bekele}: is a National Research Council Fellow at the US Naval
Research Lab (NRL) in the Navy Center for Applied Research in Artificial
Intelligence (NCARAI). Esube has extensive experience in data science kind of
competitions and he currently serves in the organizing team od DC Data Science
(Large meetup for data science professions in the Washington, DC area).
Currently, he holds a master rank at kaggle.com. Esube will serve as the
primary coordinator of the competition providing data, development kit,
platform administration, baseline methods and evaluation performance
metrics--all with in the scope of his regular work as fellow at NRL.

{\bf Joy Buolamwini}:

{\bf Timnit Gebru}:

{\bf Wallace Lawson}: is a Research Scientist at the US Naval Ressearch Lab
(NRL) in the Navy Center for Applied Research in Artificial Intelligence
(NCARAI).  He has research and published extensively on facial analysis,
including image morphing, open-set face recognition and attributes prediction.
Wallace will serve as an evaluator and will review submissions to the
competition. As mentor to Esube at NRL, Wallace also will oversee the
competition progress and will serve as an advisor.

%{Ian Goodfellow}:

%{Samy Bengio}:

\section{Resources}
\subsection{Existing resources, including prizes}

% Describe your resources (computers, support staff, equipment, sponsors, and
% available prizes).
We hope to host the competition at kaggle.com similar to last year's NIPS
competition on adversarial examples. We do not plan to provide monetary prizes.
However, top 3 submissions in each track will be invited to present in Black in
AI workshop, which will be collocated with NIPS 2018 and this competition track
of NIPS 2018.

%% The file named.bst is a bibliography style file for BibTeX 0.99c

\bibliographystyle{named}
%\bibliographystyle{apacite}
\bibliography{egbib}

\end{document}
